{\large\bf B\_CURVE} {\em a two-compartment rCBF model used for delay correction}
\begin{verbatim}


      integral = b_curve (args, shifted_g_even, ts_even, A, ... 
                          fstart, flengths)


\end{verbatim}

  Used by blood delay correction, this function implements a
  two-compartment rCBF model used for fitting the blood
  curve data to the head curve data.
\newpage

%--------------------------------------

{\large\bf CORRECTBLOOD} {\em  perform delay and dispersion corrections on blood curve}
\begin{verbatim}


   [new_ts_even, Ca_even, delta] = correctblood (A, FrameTimes, ...
                                   FrameLengths, g_even, ts_even, progress)


\end{verbatim}

   The required input parameters are: 
\begin{description}
\item \code{A} - brain activity, averaged over all gray matter in a slice.  This
           should be in units of decay / (gram-tissue * sec), and should
           just be a vector - one value per frame.
\item \code{FrameTimes} - the start time of every frame, in seconds
\item \code{FrameLengths} - the length of every frame, in seconds
\item \code{g\_even} - the (uncorrected) arterial input function, resampled at
                some *evenly spaced* time domain.  Should be in units
                of decay / (mL-blood * sec)
\item \code{ts\_even} - the time domain at which g\_even is resampled
\end{description}
 
   The returned variables are:
\begin{description}
\item \code{new\_ts\_even} - generally the same as the old time scale,
                     with some points missing from the end.
\item \code{Ca\_even} - g\_even with dispersion and delay hopefully corrected,
                 in units of decay / (mL-blood * sec).  
\item \code{delay} - the delay time (ie. shift) in seconds
\end{description}
 
   A, FrameTimes, and FrameLengths must all be vectors with the same
   number of elements (presumably the number of frames in the study).
   g\_even and ts\_even must also be vectors with the same number of
   elements, but their size should be much larger, due to the
   resampling at half-second intervals performed by resampleblood.
   
   correctblood corrects for dispersion in blood activity by
   calculating g(t) + tau * dg/dt, where tau (the dispersion time
   constant) is taken to be 4.0 seconds.
 
   It then attempts to correct for delay by fitting a theoretical blood
   curve to the observed brain activity A(t).  This curve depends
   on the parameters alpha, beta, gamma (these correspond to K1, k2,
   and V0, although for the entire slice rather than pixel-by-pixel) and
   delta (which is the delay time).  correctblood steps through a series
   of delta values (currently -5 to +10 sec), and performs a three-
   parameter fit with respect to alpha, beta, and gamma; the value of
   delta that results in the best fit is chosen as the delay time.
 
   options is an entirely optional vector meant for debugging purposes.
   If options(1) is non-zero, then correctblood will show its progress,
   by printing out the results of progressive delay-correction fits.  If 
   it is at least 2, then correctblood will also show progress graphically,
   by displaying a graph of A(t) and the fits corresponding to every
   value of delta tried.  If options(2) is zero, then no delay correction
   will be performed; if options(3) is supplied, then it will be
   used as delta to do delay correction without the time-consuming
   fitting.
\newpage

%--------------------------------------

{\large\bf FINDINTCONVO} {\em   calculate tables of the integrated convolutions commonly used}
\begin{verbatim}


    [int1,int2,int3] = findintconvo (Ca_even, ts_even, k2_lookup,...
                                     midftimes, flengths, w1[, w2[, w3]])


\end{verbatim}

  given a table of k2 values, generates tables of weighted integrals
  that commonly occur in RCBF analysis.  Namely, int\_convo is a table of
  the same size as k2\_lookup containing
 
        int ( conv (Ca(t), exp(-k2*t)) * weight )
 
  where the integration is carried out across frames.  weight is
  one of w1, w2, or w3, each of which will generally be some simple
  function of midftimes.  findintconvo will return int2 if and only if
  w2 is supplied, and int3 if and only if w3 is supplied.  w1 is 
  required, and int1 will always be returned.  Normally, the weight
  functions should be vectors with the same number of elements as
  midftimes; however, if w1 is empty then the weighting function 
  is taken to be unity.
 
  Note that in order to correctly calculate the convolution, Ca(t) must
  be resampled at evenly spaced time intervals, and this resampled blood
  activity should be passed as Ca\_even.  The times at which it is
  sampled should be passed as ts\_even.  (These can be calculated by
  resampleblood before calling findconvints.)
 
  Then, the convolution of Ca(t) and exp(-k2*t) is resampled at the
  mid-frame times (passed as midftimes) and integrated across frames
  using flengths as dt.
\newpage

%--------------------------------------

{\large\bf RCBF1} {\em a one-compartment (double-weighted integral) rCBF model.}
\begin{verbatim}


         [K1,k2] = rcbf1 (filename, slice)


\end{verbatim}

  A one-compartment rCBF model (without V0 or blood delay and 
  dispersion) implemented as a MATLAB function.  The
  compartmental equation is solved by integrating it across
  the entire study, and then weighting this integral with two
  different weights.  When these two integrals are divided by
  each other, K1 is eliminated, leaving only k2.  A lookup
  table is calculated, relating values of k2 to values of the
  integral.  From this, k2 and be calculated.  From k2, K1 is
  easily found by substitution into the original compartmental
  equation.  See the document "rCBF Analysis Using Matlab" for
  further details of both the compartmental equations
  themselves, and the method of solution.
\newpage

%--------------------------------------

{\large\bf RCBF2} {\em a two-compartment (triple-weighted integral) rCBF model.}
\begin{verbatim}


        [K1,k2,V0,delta] = rcbf2 (filename, slice)


\end{verbatim}

  rcbf2 implements the three-weighted integral method of calculating
  k2, K1, and V0 (in that order) for a particular slice.  This
  function also returns the delay value calculated for blood
  correction.  It first reads in a great mess of data (viz., the brain
  activity for every frame of the slice, frame start times and
  lengths, plasma activity, and blood sample times).  Then, a simple
  mask is created and used to filter out roughly all points outside
  the head.
  
  The actual calculations follow the procedure outlined in the
  document "RCBF Analysis Using Matlab".  Occasionally, comments in
  the source code or documentation for various functions involved in
  the analysis will refer to equations in this document.  The most
  relevant functions in this respect are rcbf2 itself, correctblood
  and findintconvos.
  
  The starting point of the three-weighted integration method is Eq.
  10 of the RCBF document.  The left hand side of this equation, rL,
  is calculated for every pixel.  Then, a lookup table relating a
  series of k2 values to the rR (the right-hand side of Eq. 10) is
  calculated.  This lookup table should have a few hundred elements,
  as calculating rR is considerably more expensive than calculating
  rL.  Since rL and rR are equal, we use the pixel-wise values of rL
  to lookup k2 for every pixel.
  
  Then, Eq. XXX is used to calculate K1.  This requires calculating
  the moderately complicated int\_activity (left hand side of Eq. XXX)
  and the extremely complicated k2\_conv\_ints (right hand side).
  However, the expression for k2\_conv\_ints appeared already in the
  numerator of rR, so we preserve that lookup table as conv\_int1 and
  use it to lookup k2\_conv\_ints.  These two long vectors (with one
  number for every pixel) are then divided to get K1.  Finally, V0 is
  calculated via Eq. YYY.
\newpage

%--------------------------------------

{\large\bf RCBFDEMO} {\em Demonstrate the RCBF blood analysis package.}
\begin{verbatim}


    rcbfdemo(slice_number, frame_number)


\end{verbatim}

  Hard-coded to use the yates\_19445 data
  file, but allows input of a slice and
  frame to display.
\newpage

%--------------------------------------

